\begin{abstract}
In the maker - movement \cite{davies2017hackerspaces}, people need online tools to exchange knowledge, share experiences and to work in interdisciplinary teams to \textit{innovate}, \textit{design} and \textit{implement solutions}.

The method of online documentation has a strong influence on how useful, learnable, available, shareable and accessible results are \cite{harcourt2016re}.

The complexity of documentation has led makers to create a variety of online communities and to experiment with different ways of documenting their results. Because documentation is often seen as tedious and time-consuming, makers are constantly seeking optimal solutions to reduce the effort needed to document their results. On the other hand, careful documentation enables makers to collaborate and share their efforts more effectively.
\todo{[add a few paragraphs  with concrete examples, such as Github, DocuBricks, Public Lab, SparkBoard, etc. There is a huge amount of stuff out there, much of it no doubt useless. Don’t waste too much time documenting it, but illustrate the diversity by listing various concrete examples]}

In this thesis, we will \todo{from briefly to students projects MISSING - a section that briefly summarizes existing tools. There are SO many} briefly survey a range of platforms developed to support makers in documenting their projects. We then analyse in more detail the pros and cons of a platform called \textit{Instructables} then we share with you the experience we did with a platform called SDGinProgress that we adapted for the sustainable development goals. In both case we discuss how authors and readers benefit from online documentation, what motivates them to share a project and – in the case of \textit{Build-in-Progress} what are the consequences of sharing a work in progress with online community. The first part of this thesis will review \textit{Instructables}, describe how the platform works and describe in details the design orientation process. The second part we will give an introduction about hackathon, the history of hackathons and its format. In the second section of this part, we will share with you our experience in the 17 hackathons and the Open Geneva festival. Then we talk talk about our experience in 17 hackathons and the two experiment of Open Geneva, the festival of innovation. And we will finish this part with writing about the classification and the potential of hackathons. The third part will review the design approach of \textit{SDG-in-Progress} and the added features. Also, we will review the user interaction of SDGinProgress.  The last part will review the user interaction of \textit{SDGinProgress} and we discuss about the limits of online platform.
\end{abstract}