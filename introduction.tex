% Some commands used in this file
\newcommand{\package}{\emph}

\chapter{Introduction}

In the maker - movement \cite{davies2017hackerspaces}, people need online tools to exchange knowledge, share experiences and to work in interdisciplinary teams to \textit{innovate}, \textit{design} and \textit{implement solutions}.

The method of online documentation has a strong influence on how useful, learnable, available, shareable and accessible results are \cite{harcourt2016re}.

The complexity of documentation has led makers to create a variety of online communities and to experiment with different ways of documenting their results. Because documentation is often seen as tedious and time-consuming, makers are constantly seeking optimal solutions to reduce the effort needed to document their results. On the other hand, careful documentation enables makers to collaborate and share their efforts more effectively.


In this state-of-the-art, we will briefly survey a range of platforms developed to support makers in documenting their projects. We then analyse in more detail the pros and cons of two platforms, \textit{Instructables} and Build-in-Progress. In both case we discuss how authors and readers benefit from online documentation, what motivates them to share a project and – in the case of \textit{Build-in-Progress} what are the consequences of sharing a work in progress with online community.

The first part of this state-of-the-art will review \textit{Instructables}, describe how the platform works and describe in details the design orientation process. The second part will review the design approach of \textit{Build-in-Progress} and the added features. The last part will review the user interaction of \textit{BiP}. 

