\begin{abstract}

The education landscape is evolving rapidly, and so are the digital tools that support it. 

A number of related trends in learning methodologies, such as challenge-based learning \cite{wiki:cbl}, hands-on learning \cite{wiki:hol}, action-based learning \cite{website:abl} and design thinking \cite{book:dorst2015frame} have established themselves as widely used alternatives to traditional classroom teaching.

In these contexts, people need online tools to exchange knowledge, share experiences and to work in interdisciplinary teams to \textit{innovate, design} and \textit{implement solutions}.

New types of learning communities, like the maker community \cite{book:makerculture}, open source hardware developers \cite{website:openhardware} and the fab lab movement \cite{book:gershenfeld2005fab}, are also setting new requirements for digital documentation tools.

Popular innovation formats such as hackathons \cite{website:openbdshackathon} and design sprints \cite{website:gvDesignSprint} are increasingly used in an educational context. In such events, time is limited and teams form and disband in days, adding additional demands for successful documentation.

The method of online documentation has a strong influence on how useful, learnable, available, shareable and accessible results are \cite{harcourt2016re}. The complexity of documenting team-based projects has led to the creation of a variety of online communities and to many experiments with different ways of documenting results. Because documentation is often seen as tedious and time-consuming, developers are constantly seeking optimal solutions to reduce the effort needed to document results, while at the same time enabling users to collaborate and share their efforts more effectively.

In the first chapter of this thesis, we briefly survey a range of platforms developed to support of team-based projects in settings ranging from public hackathons to semester-long student projects. We analyze in more detail the pros and cons of a platform called Instructable, one of the most popular in the maker community. 


In the second chapter, we introduce Build in Progress (BiP), originally developed by a researcher at MIT Media Lab for makers to document their creations in a more intuitive, visual way. We present results from developing and deploying SDG-in-Progress, a version of BiP  that we customized for students at University of Geneva working on projects related to the United Nations Sustainable Developments Goals (SDGs).


In the third chapter, we focus on two contrasting formats where such a documentations tool is needed : hackathons that last just a few days, with teams made of people from a wide range of backgrounds, and a two-month summer school, where students build stronger teams over a longer time. We describe experience in situations with many concurrent hackathon projects, such as ‘17 hackathons’ and the Open Geneva festival, and how this experience informs decisions about suitable documentation tools. 


In a concluding section we summarize our findings, and suggest next steps for developing and testing online platforms for challenge-based learning.

\end{abstract}
